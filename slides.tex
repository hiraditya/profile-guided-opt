\documentclass{beamer}
\usepackage{hyperref}
\begin{document}

\title{Profile guided optimization using open source tools}
\author{Sebastian Pop and Aditya Kumar}
\institute{SARC: Samsung Austin R\&D Center}
%\date{May 5, 2015}
\date{\today}

\frame{\titlepage}

\frame{\frametitle{Profile guided optimization (Introduction)}
    \begin{itemize}
        \item Iterative approach to program optimization.
            \begin{itemize}
                \item Compile -- Run + Collect Profile --Process Profile data -- Compile with profile info. -- Run
            \end{itemize}
        \item Intrusive approach (PGO)
        \item Non-intrusive approach (AutoFDO)
    \end{itemize}
}

\frame{\frametitle{Open source tools}
    \begin{itemize}
        \item Compiler: gcc, llvm
        \item Gprof
        \item Linux Perf
        \item Create\_gcov
        \item Gooda
    \end{itemize}
}

\frame{\frametitle{PGO: Compiler instrumentation}
  \begin{itemize}
  \item Compiler inserts probes in the code.
    \item Instrumented program collects profile as it executes.
    \item gprof/gcov used to process sampled output and display.
    \item gprof: gcc -fprofile-generate test.cpp
    \item gcov: gcc -fprofile-arcs -ftest-coverage test.cpp
    \item gprof [exe] [gmon.out] $>$ [outfile]
  \end{itemize}
}
\frame{\frametitle{AutoFDO: Non Intrusive}
    \begin{itemize}
        \item Linux perf collects profiles (Sample based)
        \item No compiler instrumentation, debug-info required while running.
        \item Negligible overhead (less than a few percent) [Google]
        \item perf record/report/stat
        \item compiling program: gcc -g3 -O2 test.cpp
        \item collecting profile: perf record -b [exe] [-o perf.data]
        \item report: perf report [perf.data]
    \end{itemize}
}


\frame{\frametitle{Challenges for non-intrusive AutoFDO}
    \begin{itemize}
        \item value based profiling.
        \item only works on recent intel machines.
        \item tradeoff between accuracy of profile and overhead.
        \item Goal is to be as close to the instrumented profile.
        \item Getting it to work requires patching the kernel.
    \end{itemize}
}

\frame{\frametitle{Comparison Intrusive vs. Non-intrusive.}
    \begin{itemize}
        \item Linux-perf: Run on different machines and at different time intervals. 
        \item Linux-perf: Good for system wide profiling.
        \item Instrumented: Gives very precise profile information.
        \item Instrumented: Good for benchmarking, analyzing small programs.
    \end{itemize}
}


\frame{\frametitle{create\_gcov}
    \begin{itemize}
        \item Creates coverage file from perf.data to be used by the compiler.
        \item optimize with profile information: gcc -fauto-profile=file.gcov -O2 test.cpp
        \item Source: https://github.com/google/autofdo
     \end{itemize}
}

\frame{\frametitle{gooda}
    \begin{itemize}
        \item gooda-analyzer
        \item gooda-visualizer
        \item Source: https://github.com/David-Levinthal/gooda
        \item Requires patching linux perf to link against libpfm4 [See: gooda-analyzer/README] 
        \item Requires LBR (Available in Intel Sandy Bridge, Ivy Bridge, Westmere)
    \end{itemize}
}

\frame{\frametitle{References}
    \begin{itemize}
    \item \url{http://static.googleusercontent.com/media/research.google.com/en//pubs/archive/36575.pdf}
    \item \url{http://www.burningcutlery.com/derek/docs/instant-profiling-CGO13.pdf}
    \item \url{https://github.com/David-Levinthal/gooda}
    \item \url{https://github.com/google/autofdo}
    \end{itemize}
}

\end{document}
